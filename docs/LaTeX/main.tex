% Auth: Nicklas Vraa
% Docs: https://github.com/NicklasVraa/LiX

\documentclass{ieee}
\usepackage[all, stdclass]{lix}
\usepackage{amsmath}
\usepackage{accents}
\newlength{\dhatheight}
\newcommand{\doublehat}[1]{%
    \settoheight{\dhatheight}{\ensuremath{\hat{#1}}}%
    \addtolength{\dhatheight}{-0.35ex}%
    \hat{\vphantom{\rule{1pt}{\dhatheight}}%
    \smash{\hat{#1}}}}
\size     {letter}
% \header   {LiX IEEE Journal Template for \LaTeX}
\title    {Padding Oracle Attack}
\authors  {Mohammad Abu-Shelbaia \\
1200198}
% \keywords {Research, templates, metapackages, packages,
           % LaTeX, IEEE, LiX, Simplicity, Abstraction}
% \idnum    {1200198}

\abstract{
The "Padding Oracle Attack Lab" is a hands-on practical exercise that exploits a vulnerability in a specific encryption algorithm. This lab focuses on demonstrating the weakness of the padding scheme employed by the encryption algorithm, which can lead to a successful attack known as the padding oracle attack.}

\begin{document}

\h{Introduction}
\l{P}adding Oracle Attack is an attack that uses the padding validation of an encrypted message to decrypt the ciphertext. Varying message lengths get padded to be compatible with the block cipher encryption algorithm, the attack depends on the Oracle server which responds to queries about whether a message is padded correctly or not. The attack is mostly associated with CBC mode decryption used with block ciphers.\cite{wiki-padding-oracle}
\hh{Cipher Block Chaining (CBC)}
block cipher mode of operation that provides information security such as confidentiality or authenticity. It is an algorithm that uses a block cipher to securely transform amounts of data larger than a block. In CBC mode, each block of plaintext is XORed with the previous ciphertext block before being encrypted. This way, each ciphertext block depends on all plaintext blocks processed up to that point.\cite{wiki-cbc}
\fig{cbc_diagram}{1}{assets/d_diagram.png}
{CBC Decryption Diagram}
\hh{PKCS5 Padding}
If the size of each block is denoted as B, the process involves adding N padding bytes to the input to reach the next multiple of B, taking into account the following conditions. When the input length is not an exact multiple of B, N padding bytes, each having the value of N, are appended. However, if the input length is already a multiple of B, B bytes of value B are added. This ensures that padding, ranging from one to B bytes in length, is consistently applied without any ambiguity. Once the decryption is complete, a verification step is performed by checking that the last N bytes of the decrypted data all have the value N, where N is greater than 1 but not exceeding B. If this condition is met, N bytes are removed; otherwise, a decryption error is thrown.\cite{cryptosys}
\h{Procedure}
\hh{Getting Familiar With Padding}
This task is a demonstration of how PKCS5 Padding works, for this task we will encrypt a message of variable lengths using AES-128, and decrypt them with `-nopad` flag to observe the appended padding. We can run the \c{Makefile} using \c{make run} command.

\code{my_code}{Bash}{
    run:
        @echo "[5 Bytes] 12345"
        @echo -n "12345" | openssl enc -aes-128-cbc -e -k hi -md sha256 -pbkdf2 | openssl enc -aes-128-cbc -d -nopad -k hi -md sha256 -pbkdf2 | xxd

        @echo "\n[10 Bytes] 0123456789"
        @echo -n "0123456789" | openssl enc -aes-128-cbc -e -k hi -md sha256 -pbkdf2 | openssl enc -aes-128-cbc -d -nopad -k hi -md sha256 -pbkdf2 | xxd

        @echo "\n[16 Bytes] 0123456789ABCDEF"
        @echo -n "0123456789ABCDEF" | openssl enc -aes-128-cbc -e -k hi -md sha256 -pbkdf2 | openssl enc -aes-128-cbc -d -nopad -k hi -md sha256 -pbkdf2 | xxd
}
{Padding Task}
\fig{task1_picture}{1}{assets/task1.png}
{Padding Output}
As we can see in Figure \ref{task1_picture}, when a message has a size of 5 bytes, 11 bytes of padding are appended, each with a value of '0x0b' (equivalent to 11 in decimal). For a 10-byte message, 6 bytes of padding are appended, each with a value of '0x06'. Lastly, for a message with a size of 16 bytes, a full block of padding is added, with each byte having a value of '0x10' (equivalent to 16 in decimal), as 16 bytes is a multiple of the block size.
\hh{Padding Oracle Attack (Level 1)}
In this task, we are going to decrypt the last few Bytes of the plain text manually.
To get the last byte of the second block we have to change the last byte of the first cipher-text, since \math{my_equation}{P_{32} = C_{16} \oplus D_{32}} we will loop through all the values of $C_{16}$ until we land at a value at which we get a MAC error instated of a padding error. Knowing the value that gives us the correct padding we can decrypt the plain text according to the following equation: 
\math{eq}{
    \hat{P_{32}} = \hat{C_{16}} \oplus D_{32}
}
where $\hat{P_{32}}$ is the value at which we get a correct padding, and since we know that \math{eq2}{D_{32} = C_{16} \oplus P_{32}} we can show that 
\math{eq3}{
    P_{32} = \hat{P_{32}} \oplus C_{16} \oplus \hat{C_{16}}
}

\code{my_code}{python}{
    oracle = PaddingOracle('10.9.0.80', 5000)
    IV, C1, C2 = oracle.ctext[:16], oracle.ctext[16:32], oracle.ctext[32:48]
    C1_copy, IV_copy = bytearray(16), bytearray(16)
    P = bytearray(32)
    for i in range(2**8): 
        C1_copy[15] = i
        status = oracle.decrypt(IV + C1_copy + C2)
        if status == "Valid":
            break
    P[31] = 1 ^ C1_copy[15] ^ C1[15]
    print(f"P[31]: {hex(P[31])}")
    print(f"C1'[15]: {hex(C1_copy(15)}")
    # Output:
    #    P[31]: 0x3
    #    C1'[15]: 0xcf
}
{Decrypting the last byte of the plain text}
To get the second to last byte we need to adjust the padding of the first 2-bytes, looping through all possible values of the deemed impossible with the complexity arising exponentially, so we will use another way which is to manipulate the last byte so we can repeat previous steps. First, we need to manipulate the output of the last byte to 0x02, so the server would go and check the second-byte padding we can do so by manipulating the last byte of the first block as shown below:
\math{eq}{
\doublehat{C_{16}} = P_{32} \oplus C_{8} \oplus \doublehat{P_{32}}
}
where $\doublehat{P_{32}}$ is the value we want present in $P_{32}$.

\code{my_code}{python}{

    oracle = PaddingOracle('10.9.0.80', 5000)
    IV, C1, C2 = oracle.ctext[:16], oracle.ctext[16:32], oracle.ctext[32:48]
    C1_copy, IV_copy = bytearray(16), bytearray(16)
    P = bytearray(32)
    P[31] = 0x03
    C1_copy[15] = C1[15] ^ P[31] ^ 0x02
    for i in range(2**8): 
        C1_copy[14] = i
        status = oracle.decrypt(IV + C1_copy + C2)
        if status == "Valid":
            break
    P[30] = 0x02 ^ C1_copy[14] ^ C1[14]
    print(f"P[30]: {hex(P[30])}")
    print(f"C1'[14]: {hex(C1_copy[14])}")
    # Output:
    #    P[30]: 0x3
    #    C1'[14]: 0x39
}
{Decrypting the second to last byte of the plain text}
\fig{my_fig}{1}{assets/task2.png}
{Decrypting the last two bytes from the plain text}

In order to expedite and automate the process of decrypting the values, manual byte-by-byte manipulation of the ciphertext is indeed impractical and time-consuming, especially for large messages with numerous blocks. Therefore, an automated approach becomes necessary to handle the decryption of larger ciphertexts. In the next section, we will explore a more efficient method to automate the process, enabling us to decrypt the plaintext values effectively and save considerable time and effort.

\hh{Padding Oracle Attack (Level 2)}
In this part, we wrote a small script to send queries to the server automatically according to the previous responses, we ran the program on both servers \c{10.9.0.80:5000} and \c{10.9.0.80:6000}, and then compared the result we got with the provided plain text.

\code{my_code}{python}{
    oracle = PaddingOracle('10.9.0.80', 5000)
    IV, C1, C2 = oracle.ctext[:16], oracle.ctext[16:32], oracle.ctext[32:48]
    P = bytearray(32)
    C1_copy, IV_copy = bytearray(16), bytearray(16)
    for k in range(1, 17):
        for i in range(256):
            C1_copy[16 - k] = i
            status = oracle.decrypt(IV+ C1_copy + C2)
            if status == "Valid":
                break
    
        P[32 - k] = C1_copy[16 - k] ^ C1[16 - k] ^ k
        for j in range(1, k+1):
            C1_copy[16 - j] = C1[16 - j] ^ (k + 1) ^ P[32 - j]
        
        for i in range(256):
            IV_copy[16 - k] = i
            status = oracle.decrypt(IV_copy + C1)
            if status == "Valid":
                break
                
        P[16 - k] = IV_copy[16 - k] ^ IV[16 - k] ^ k
        for j in range(1, k+1):
            IV_copy[16 - j] = IV[16 - j] ^ (k + 1) ^ P[16 - j]
}
{Decrypting the plain text automatically}
A small notice to decrypt the first block, since we don't have any previous cipher block we change the values of the IV instated.
\fig{my_fig11}{1}{assets/task3.1.png}
{Plain Text retrieved from the server on port 5000}
\fig{my_fig22}{1}{assets/task3.2.png}
{Plain Text retrieved from the server on port 6000}
\fig{my_fig3}{1}{assets/task3.3.png}
{Server Status}

Comparing the results in (Figure: \ref{my_fig11}) and the provided plain text, we conclude that our script is running correctly, and by referring to (Figure: \ref{my_fig3}) we can see that we retrieved the plain text with one connection to the server.



    % \tabs{my_table}{cols}{
    % This & is & a & cool & table \\
    % 1    & 2  & 3 & 4    & 5     \\
    % a    & b  & c & d    & e
    % }
    % {This is a table - Notice that the description wraps.}

    % \begin{bullets}
    %     \item This is a very long item to test the wrapping of text in the unordered environment.
    %     \begin{bullets}
    %         \item Another item, but indented.
    %         \begin{bullets}
    %             \item Yet another item.
    %             \item An item on the same level.
    %         \end{bullets}
    %     \end{bullets}
    % \end{bullets}

    % \begin{numbers}
    %     \item This is an item.
    %     \begin{numbers}
    %         \item Another item, but indented.
    %         \begin{numbers}
    %             \item Yet another item.
    %             \item An item on the same level.
    %         \end{numbers}
    %     \end{numbers}
    %     \item Last item.
    % \end{numbers}


% \h{Referencing}
% Here is a hyperlink to a \url{webpage}{https://www.overleaf.com/}. This is a reference to \r{my_equation}, and this refers to \r{my_svg}. You can also refer to headings like \r{Tables}. \R{my_code} is a capitalized variant. For all references, both the name and number are links. Of cource, you can cite sources using \c{\cite{...}}, like this \cite{minted} or this \cite{tabularray}. You can also insert a bibliography.

\h{Conclusion}
In this lab, we explored the Padding Oracle Attack, which exploits vulnerabilities in the padding scheme of an encryption algorithm. By leveraging the padding validation responses from the Oracle server, we were able to decrypt the ciphertext. Through manual decryption of the last few bytes of the plaintext and an automated approach using a script, we successfully recovered the original message. This exercise highlights the importance of secure padding schemes and the potential risks associated with inadequate implementations. By understanding the weaknesses exposed by the Padding Oracle Attack, we can strengthen encryption protocols to enhance overall data security.
\bib{cite.bib}{IEEEtran}
\end{document}
